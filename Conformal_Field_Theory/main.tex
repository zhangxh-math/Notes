\documentclass[10pt,reqno,draft]{article}

\usepackage{amsmath,amsfonts,amssymb,amsthm,version}
\usepackage{mathrsfs,fancybox,pifont}
\usepackage{graphicx}
\usepackage{url,hyperref}
\usepackage[notcite,notref]{showkeys}
\usepackage{color}
\usepackage{subfigure,multirow}
\usepackage{epstopdf}
\usepackage{cases}
\usepackage[shortlabels]{enumitem}
\usepackage{mathtools}
\usepackage{algorithm,algorithmic}
\usepackage{authblk}
%\usepackage{fancyhdr}
\usepackage{lipsum}
\usepackage{tikz-cd}
%\usepackage{MnSymbol}
\usepackage{braket}

\usepackage{import}
\usepackage{xifthen}
\usepackage{pdfpages}
\usepackage{transparent}

\newcommand{\incfig}[1]{%
    \def\svgscale{0.38}
    \import{./Figures/}{#1.pdf_tex}
}

\allowdisplaybreaks

\setlength{\textwidth}{15cm}
\setlength{\textheight}{21.6cm}
\setlength{\oddsidemargin}{.5cm}
\setlength{\evensidemargin}{.5cm}
\baselineskip 13pt

%\renewcommand{\baselinestretch}{1.1}
\renewcommand {\thefootnote}{\fnsymbol{footnote}}

%Notation
\newcommand{\dd}{\mathop{}\!\mathrm{d}}

% number of equation, figure and table
\numberwithin{equation}{section}
\numberwithin{figure}{section}
\numberwithin{table}{section}

% command for equations, theorems and lemmas etc.
\theoremstyle{plain}
\newtheorem{theorem}{Theorem}[section]
\newtheorem{lemma}[theorem]{Lemma}
\newtheorem{corollary}{Corollary}[theorem]
\newtheorem{proposition}[theorem]{Proposition}

\theoremstyle{definition}
\newtheorem{definition}[theorem]{Definition}
\newtheorem{example}[theorem]{Example}
\newtheorem{exercise}[theorem]{Exercise}

\theoremstyle{remark}
\newtheorem{remark}{Remark}[theorem]

\title{Conformal Field Theory}
\author[1]{Xuhui Zhang\thanks{zhangxh.math@gmail.com}}
%\author[1]{JIAN ZHOU\thanks{jianzhou@mail.tsinghua.edu.cn}}
%\author[2]{Author C\thanks{C.C@university.edu}}
%\author[2]{Author D\thanks{D.D@university.edu}}
\affil[1]{Department of Mathematical Science, Tsinghua University}
%\affil[2]{Department of Mechanical Engineering, \LaTeX\ University}

% Corresponding author. {\it Email address:} xyz@math.univ.edu.

%\author{D. Doe, P. T. Frank, and J. E. Smith\thanks{ {\it Email address:} xyz@math.univ.edu.}\
%\small Department of Mathematics, XXX University,\
%\small City 123456, China}

\date{\today}

\begin{document}
    \maketitle
	This is a note of \cite{gaberdielIntroductionConformalField2000}, and we focus on the 2D conformal theory.
	\section{General structure of a local conformal theory}
	\subsection{The space of states}
	Generally speaking, the space of states is a Hilbert space \(\mathcal{H}\) and the correlation functions are defined for some dense subspace \(\mathcal{F}\) of \(\mathcal{H}\). Let \(\Omega\) be the vacuum state in \(\mathcal{F}\) and \((-,-)\) be the inner product of \(\mathcal{H}\). A 2D conformal field theory is defined on a Riemann surface with the coordinate \(z\), we will assign an smooth operator field \(V(\psi,z)\) to each state \(\psi\in \mathcal{F}\) such that 
	\begin{equation}
	  \psi=V(\psi,z_{0})\Omega 
	\end{equation}
	for some \(z_0\) which we assume to be 0 usually. The correlation function is defined by
	\begin{equation}
		\braket{V(\psi_1,z_1)\cdots V(\psi_{n},z_{n})}:=(\Omega,V(\psi_1,z_1)\cdots V(\psi_{n},z_{n})\Omega).
	\end{equation}
	The point \(z_0\) is called the "past infinity" in physics. If we fix a future infinity \(z_{\infty}\), we will get some sense of time evolution. 

	Given a state \(\psi\in \mathcal{F}\), if for any state \(\psi_{i}\in \mathcal{F}\), the correlation functions 
	\begin{equation}
		\braket{V(\psi,z)V(\psi_1,z_1)\cdots V(\psi_{n},z_{n})}
	\end{equation}
	is a meromorphic function, then \(\psi\) is called a meromorphic state. The space \(\mathcal{F}_{0}\) consisting of all meromorphic states is called a meromorphic subtheory of \(\mathcal{H}\). Similarly, there is an anti-meromorphic subtheory \(\overline{\mathcal{F}}_{0}\) with a similar definition. 

	The correlation functions of the theory determine the \emph{operator product expansion}(OPE)
	\begin{equation}\label{eq:OPE}
	  V(\psi_{i},z_1)V(\psi_2,z_2)=\sum_{i} (z_1-z_2)^{\Delta_{i}}(\bar{z}_{1}-\bar{z}_{2})^{\overline{\Delta}_{i}}\sum_{r,s\ge 0} V(\phi_{r,s}^{i},z_2)(z_1-z_2)^{r}(\bar{z}_{1}-\bar{z}_{2})^{s},
	\end{equation}
	where \(\Delta_{i},\overline{\Delta}_{i}\in \mathbb{R}\) and \(r,s \in \mathbb{N}\) and \(\psi_1,\psi_2,\phi_{r,s}^{i}\in \mathcal{F}\). The OPE \eqref{eq:OPE} defined so-called a vertex operator algebra of meromorphic fields and anti-meromorphic fields in \cite{borcherdsMonstrousMoonshineMonstrous1992,borcherdsVertexAlgebrasKacMoody1986}. The OPE can be read off from 
	\begin{equation}
		\begin{aligned}
			&\braket{ V(\psi_{i},z_1)V(\psi_2,z_2)V(\phi_1,w_1)\cdots V(\phi_n,w_{n})}\\
			=&\sum_{i} (z_1-z_2)^{\Delta_{i}}(\bar{z}_{1}-\bar{z}_{2})^{\overline{\Delta}_{i}}\sum_{r,s\ge 0} (z_1-z_2)^{r}(\bar{z}_{1}-\bar{z}_{2})^{s}\braket{V(\phi_{r,s}^{i},z_2)V(\psi_1,w_1)\cdots V(\psi_{n},w_{n})}.
		\end{aligned}
	\end{equation}
	Since the OPE is associative,  \(\mathcal{F}\) is a representation of two vertex operator algebras. \(\mathcal{F}\) can be decomposed into indecomposable representations 
	\begin{equation}\label{eq:indecomposable}
	  \mathcal{H}=\bigoplus_{(j,\bar{j})}\mathcal{H}_{(j,\bar{j})}.
	\end{equation}
	If the decomposition \eqref{eq:indecomposable} is finite, then we call this theory a finite theory.
	\subsection{Modular invariance}
	To introduce the modular invariance, we consider torus, i.e. the Riemann surface of genus 1. Recall that complex structures of torus is parametrized by \(\tau\in \{z\in \mathbb{C}|\mathrm{Im} z>0\} \) modular the action 
	\begin{equation}
	  \tau\mapsto A\tau=\frac{a\tau+b}{c\tau+d},
	\end{equation}
	where
	\begin{equation}
	  A=\begin{pmatrix}
		  a & b \\
		  c & d	 
	  \end{pmatrix}
	  \in \mathcal{M}:=SL(2,\mathbb{Z})/\mathbb{Z}_{2}.
	\end{equation}
	If we cut the torus along a non-trivial cycle, we can obtain a annulus, on which there is a propagator along the annulus:
	\begin{equation}
		\mathcal{O}(q,\bar{q})=q^{L_{0}-\frac{c}{24}}\bar{q}^{\bar{L}_{0}-\frac{\bar{c}}{24}}
	\end{equation}
	The vacuum correlator(or the partition function) on the torus is describe by the trace of \(\mathcal{O}(q,\bar{q})\):
	\begin{equation}
		\sum_{(j,\bar{j})} \operatorname{Tr}_{\mathcal{H}_{(j,\bar{j})}}\mathcal{O}(q,\bar{q}),
	\end{equation}
	where \(L_0\) and \(\bar{L}_0\) are the scaling operators of two vertex algebras and \(c\) and \(\bar{c}\) their central charges. In our case of torus, \(q=e^{2\pi i\tau}\). Since \(\tau\) and \(A\tau\) describe the same torus, the partition function is invariant under the transformation \(\tau\mapsto A\tau\).

	In most of cases, 
	\begin{equation}
	  \mathcal{H}_{(j,\bar{j})}=\mathcal{H}_{j}\otimes\mathcal{H}_{\bar{j}}
	\end{equation}
	where \(\mathcal{H}_{j}\) is an irreducible representation of meromorphic vertex operator algebra and \(\mathcal{H}_{\bar{j}}\) is an irreducible representation of anti-meromorphic vertex operator algebra. The partition function on the torus is given by 
	\begin{equation}
	  \sum_{(j,\bar{j})} \chi_{j}(\tau)\bar{\chi}_{\bar{j}}(\bar{\tau}),
	\end{equation}
	where \(\chi_{j}(\tau)=\operatorname{Tr}_{\mathcal{H}_{j}}(q^{L_0-\frac{c}{24}}) \) and the \(\bar{\chi}_{\bar{j}}(\bar{\tau})\) is defined by a similar formula. A remarkable fact about a big class of vertex operator algebras is that characters transform into one another under the modular group \(\mathcal{M}=SL(2,\mathbb{Z})/\mathbb{Z}_{2}\)
	\begin{equation}
		\chi_{j} \left( -\frac{1}{\tau} \right) =\sum_{k} S_{j}^{k}\chi_{k}(\tau),\quad \chi_{j}(\tau+1)=\sum_{k} T_{j}^{k}\chi_{k}(\tau).
	\end{equation}
	And we define \(\bar{S}_{\bar{j}}^{\bar{k}}\) and \(\bar{T}_{\bar{j}}^{\bar{k}}\) by a similar method for anti-meromorphic vertex operator algebra. Assume that 
	\begin{equation}
	  \mathcal{H}=\bigoplus_{i,\bar{j}}M^{i\bar{j}}\mathcal{H}_{i}\otimes\bar{\mathcal{H}}_{\bar{j}}.
	\end{equation}
	Then we have 
	\begin{equation}
	  \sum_{i,\bar{j}} S_{i}^{l}M^{i,\bar{j}}\bar{S}_{\bar{j}}^{\bar{k}}=\sum_{i,\bar{j}} T_{i}^{l}M^{i,\bar{j}}\bar{T}_{\bar{j}}^{\bar{k}}=M^{l\bar{k}}.
	\end{equation}
	This provides powerful constrains for matrix \(M^{i\bar{j}}\). In the case of finite theory, these conditions allow one to obtain finite solution and a nice classification. See \cite{cappelliModularInvariantPartition1987,cappelliADEClassificationMinimal1987,gannonCappelliItzyksonZuberADEClassification2000} and \cite{gannonClassificationSUModular,gannonModularInvariantsMinimal1997}.

	\section{Meromorphic conformal field theory on the sphere}
	In this section, our main objects are the meromorphic fields in \(\mathcal{F}_{0}\). For each meromorphic field \(\psi\), there is a vertex operator \(V(\psi,z)\) which create \(\psi\) from the vacuum state \(\Omega\). The operators are assume to be local in the sense that 
	\begin{equation}
	  V(\psi,z)V(\phi,w)=\varepsilon V(\phi,w)V(\psi,z)
	\end{equation}
	for \(z\neq w\). If \(\varepsilon=-2\), then \(\psi\) and \(\phi\) are both fermionic and \(\varepsilon=1\) otherwise. The meromorphic states space \(\mathcal{F}_{0}\) can be decomposed as 
	\begin{equation}
	  \mathcal{F}_{0}=\mathcal{F}_{0}^{B}\oplus\mathcal{F}_{0}^{F},
	\end{equation}
	where \(\mathcal{F}_{0}^{B}\) is the space of bosonic states and \(\mathcal{F}_{0}^{F}\) is the space of fermionic states. In the following text, we assume that a state is either fermionic or bosonic. 

	We consider the transformation of states under M\"{o}bius group \(\mathcal{M}\). The generators of \(\mathcal{M}\) are 
	\begin{equation}
	  e^{\lambda\mathcal{L}_{-1}}(z)=z+\lambda,\quad e^{\lambda\mathcal{L}_{0}}(z)=e^{\lambda}z,\quad e^{\lambda\mathcal{L}_{1}}(z)=\frac{z}{1-\lambda z}.
	\end{equation}
	In the language of \(SL(2,\mathbb{C})\), we have 
	\begin{equation}
	  e^{\lambda\mathcal{L}_{-1}}=\begin{pmatrix}
	    1 & \lambda\\
	    0 & 1
	  \end{pmatrix},\quad e^{\lambda\mathcal{L}_{0}}=\begin{pmatrix}
	    e^{\frac{\lambda}{2}} &0\\
	    0 & e^{-\frac{\lambda}{2}}
	  \end{pmatrix},\quad e^{\lambda\mathcal{L}_{1}}=\begin{pmatrix}
	    1&0\\
	    -\lambda&1
	  \end{pmatrix} .
	\end{equation}
	And the Lie algebra of \(\mathcal{M}\) is generated by 
	\begin{equation}
	  \mathcal{L}_{-1}=\begin{pmatrix}
	    0 & 1\\
	    0 & 0
	  \end{pmatrix} ,\quad\mathcal{L}_{0}=\begin{pmatrix}
	    \frac{\lambda}{2} & 0\\
	    0 & -\frac{\lambda}{2}
	  \end{pmatrix} ,\quad\mathcal{L}_{1}=\begin{pmatrix}
	    0 & 0\\
	    -1 & 0
	  \end{pmatrix} .
	\end{equation}
	They satisfy the commutation relations
	\begin{equation}
	  [\mathcal{L}_{m},\mathcal{L}_{n}]=(m-n)\mathcal{L}_{m+n},\quad m,n=0,\pm 1. 
	\end{equation}
	
	In physics, \(\mathcal{F}_{0}\) forms a representation of this algebra, and we associate the operator \(L_{m}\) to \(\mathcal{L}_{m}\). \(L_{0}\) can be identified with the energy operator, hence the spectrum of \(L_{0}\) is bounded from below. \(\mathcal{F}_{0}\) can be decomposed into irreducible representations, in a given irreducible highest weight representation, we denote by \(\psi\) the highest weight vector with weight \(h\). Since 
	\begin{equation}
	  L_{0}L_{1}\psi=\left( h-1 \right) L_{1}\psi,
	\end{equation}
	we have \(L_{1}\psi=0\). States with property 
	\begin{equation}
	  L_1\psi=0,\quad L_0\psi=h\psi
	\end{equation}
	are called \emph{quasiprimary}, and \(h\) is called the conformal weight of \(\psi\). Each quasiprimary state \(\psi\) generate a irreducible representation of \(sl(2,\mathbb{C})\) that consists of \(L_{-1}\)-descendants of \(\psi\). Notice that 
	\begin{equation}
	  L_1L_{-1}^{n}\psi=2n\left( h+\frac{1}{2}(n-1) \right) L_{-1}^{n-1}\psi,
	\end{equation}
	if \(h\) is an half integer, the above representation is given by \(\operatorname{ Span }_{\mathbb{C}}\{L_{-1}^{n}\psi|n\ge 0\} \) moduli the subrepresentation \(\operatorname{ Span } _{\mathbb{C}}\{L_{-1}^{n}\psi|n\ge 1-2h\} \). And we obtain a finite-dimensional irreducible representation. Then we have
	\begin{equation}
	  L_{-1}\Omega=L_{0}\Omega=L_1\Omega=0.
	\end{equation}

	Next, we consider the action of \(\mathcal{M}\) on correlation functions 
	\begin{equation}
	  \left<\prod_{i=1}^{n} V\left( \psi_{i},z_{i} \right)   \right>
	\end{equation}
	where \(\psi_{i}\) are all quasiprimary states with conformal weight \(h_{i}\). The action of \(\gamma\in \mathcal{M}\) on it is defined by 
	\begin{equation}
	  \left<\prod_{i=1}^{n} V\left( \psi_{i},z_{i} \right)   \right>=\prod_{i=1}^{n} \left( \frac{\dd \gamma\left( z_{i} \right) }{\dd z_{i}} \right)^{h_{i}} \left<\prod_{i=1}^{n} V\left( \psi_{i},\gamma\left( z_{i} \right) \right)   \right>
	\end{equation}
  
  \subsection{Some examples}
  \subsubsection{The free boson}
  Here we consider a single free boson. By factorization, the states space reduces to a vector space \(V\) of dimension 1. Suppose it is generated by \(J\) of conformal weight 1, and the corresponding vertex operator is 
  \begin{equation}
    J(z):=V(J,z)=\sum_{n\in \mathbb{Z}} J_{n}z^{-n-1}.
  \end{equation}
  The Fock space is given by the linear combination of 
  \begin{equation}
    J(z_1)J(z_2)\cdots J(z_{m})\Omega.
  \end{equation}
  The amplitudes is given by
  \begin{equation}
    \left< J(z_1)J(z_2)\cdots J(z_{2n}) \right> = k^{n}\sum_{\pi\in S_{n}'} \prod_{j=1}^{n} \frac{1}{(z_{\pi(j)}-z_{\pi_{j+n}})^{2}},
  \end{equation}
  where \(S_{n}'\) is a subgroup of \(S_{n}\) defined by
  \begin{equation}
    S_{n}'=\{\sigma=S_{n}|\sigma(1)<\sigma(3)<\cdots<\sigma(2n-1),\sigma(2k-1)<\sigma(2k)\forall k=1,\ldots,n\} .
  \end{equation}
  And the amplitudes with odd number of \(J(z)\) is 0. We emphasize that this does not mean \(\left<V(\psi,z) \right> =0\) for any \(\psi\in \mathcal{F}_{0}\).

  When \(n=1 \), we have 
  \[
  \left<J(z)J(w) \right> = \frac{k}{(z-w)^2}
  .\] 
  OPE tells us that 
  \begin{equation}
    J(z)J(w)=\sum_{n\le 1} V(J_{n}J,w)(z-w)^{-n-1}. 
  \end{equation}
  Recall that \(L_0J_{n}J=(1-n)J_{n}J\), so \(J_0J=cJ\) for some \(c\in \mathbb{C}\) and \(J_1J=k\Omega\), hence
  \[
      \left<V(J_0J,w) \right> =0, \quad\left< V(J_1J,w) \right> = k.
  \]
  Hence we have 
  \begin{equation}
    J(z)J(w)\sim \frac{k}{(z-w)^2}.
  \end{equation}
  The symbol \(\sim \) means that the both sides are equal up to a non-singular part at \(z=w\).
  
	\addcontentsline{toc}{section}{References}
    \bibliographystyle{alpha}
    \bibliography{ref}
\end{document}
